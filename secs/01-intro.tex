% !TEX root = ../foresight.tex

\section{Introduction}

Introduction: Lots of accidents due to not seeing what's going on. For example a pedestrian or a biker behind a car. Could be similar to the motivation in Wilko's paper.

Autonomous systems are becoming increasingly integrated into
our daily lives. Two of the most conspicuous examples are UAVs,
which are used to capture aerial footage, and autonomous cars,
which promise to make driving much safer and more convenient
than it is today. Unfortunately, both of these types of autonomous
robot have deficiencies. UAVs are highly mobile and agile, but they
have limited range and payload capabilities. Meanwhile, cars and trucks
have range and payload capabilities several orders of magnitude greater than
those of UAVs but, being constrained to traveling on roads, are far less mobile.

In the context of autonomous cars, this mobility constraint becomes apparent
when navigating corners and crowded environments. Since a car cannot
see around corners and other obstacles without physically reorienting its
entire bulk, it must often navigate with little information about what obstacles
it may encounter. Therefore it is blind to obstacles such as a biker or little old
lady crossing the street behind an oncoming bus and therefore must proceed
with extreme caution. Pairing an autonomous car with a quadcopter is a potential
solution to this problem - the quadcopter could use the car as a charging base,
while the car could send the quadcopter out on missions to scout ahead and
fill in the blind spots in its vision.

The one crucial step in enabling a quadcopter and an autonomous car to work
together is to ensure that there exists an accurate transform between the car
and the quadcopter. GPS, which has an accuracy of about +/- 2m, can suffice
for when the quadcopter is navigating broad open spaces. However, +/- 2m
accuracy is completely insufficient for when the quadcopter must navigate
obstacle-filled environments (such as a city center) or land on the car. In this 
paper, we present a method using ultra-wideband radios to measure the relative 
position of the quadcopter relative to the car with +/- 10cm accuracy, enabling 
the quadcopter to safely navigate to blind spots and then land back on the car.

This method presents a method augment what a self-driving car can see.
- Occlusions limit the field of view of the car
- A small UAV can fly to explore those blind spots and prevent accidents
- Informative / exploration planning algorithm for the UAV
- Open source communication framework for multi-robot systems
- Experiments with real data captured from our car 

\subsection{Related works}

Motion planning for autonomous cars. Some methods assume full knowledge of the environment [] -> not realistic, while others [] only have local sensing. Our method would work in parallel and provide a "better" map.

Multi-robot mapping -> Vijay Kumar, Scaramuzza and others. Drone + ground robot.

Informative planning and exploration -> V. Kumar, D. Rus, R. Siegwart.

\subsection{Contribution}

\begin{itemize}
\item
Method for finding the relative transform between two mobile robots
\item
Method for informative sensing
\item
Experiments that prove that the quadcopter can accurately navigate
to blind spots and land back on the car.
\end{itemize}

We validate our approach with data collected with our self-driving vehicle and a small UAV.