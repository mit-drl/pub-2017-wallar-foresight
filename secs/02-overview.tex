% !TEX root = ../foresight.tex

\section{Method overview}

We consider two vehicles.
\begin{itemize}
\item
A ground vehicle - the car -, which can create a local map of the environment, localize with respect to it and autonomously navigate \cite{Add citation to Felix's IV paper}. We utilize a 2D grid map to represent the free space and obstacles seen by the car. In particular, our vehicle is equipped with a 2D LIDAR. 
\item
A lightweight companion quadrotor \cite{Parrot Bebop 2} equipped with a front facing camera. The drone is able to fly autonomously to/from the ground vehicle and detect obstacles that were originally occluded for the ground vehicle.
\end{itemize}

Given a laser scan from the ground vehicle, our objective is to:
a) determine which areas of the environment are occluded to ground vehicle,
b) compute a safe path for the aerial vehicle to observe the occluded areas, and
c) detect unseen obstacles, such as pedestrians, and report them back to the ground vehicle.

% Our problem is two pronged. First we need to be able to accurately localize the
% quadrotor in the same reference frame as the vehicle. Second, we need to
% determine which areas of the environment are occluded to ground vehicle, and
% compute a safe path that leads the quadrotor to view these regions.

% % \begin{problem}
% %
% %     Given a laser scan from the ground vehicle, compute a path the maximizes
% %     the occluded area observed by the quadrotor for a given time horizon.
% % \end{problem}
%
% We used a Parrot Bebop 2 as the quadcopter. It is equipped with a camera, and
% velocity and altitude measurements can be read from its SDK at a rate of 5 Hz.
% The Bebop uses a software gimbal to stabilize the camera image.
%
% The inputs to the Bebop are roll angle $\phi$, pitch angle $\theta$, 
% yaw rate $\omega$, and z velocity $\dot{z}$. We use
% PID controllers to control the inputs.
%
% $$ u = [\phi, \theta, \omega, \dot{z}] $$

%\subsection{Method overview}

To accurately localize the quadrotor relative to the ground vehicle, we equip the ground vehicle with several Ultra-wideband radios. In the quadrotor, we fuse, via an Unscented Kalman filter (UKF) relative information from a UWB radio with odometry estimates from a down facing optical flow sensor and an onboard IMU.
% We use an unscented Kalman filter to fuse these
% measurements and produce an estimate of quads 3D position relative to the
% ground vehicle.

Our algorithm operates directly on the laser scan from the ground vehicle to find occluded
regions.
We then employ an anytime sampling-based algorithm to compute a collision free path for the drone
that maximizes the occluded area viewed by the quadrotor.
% and builds a search tree from the quadrotor's current configuration
% towards these regions. The search terminates when a timeout has expired and
% returns the path that currently maximizes the occluded area viewed by the
% quadrotor.
To detect obstacles within the occluded areas, we employ a real-time object detecting convolutional neural network~\cite{yolo}, which is able to classify and locate objects, such as pedestrians, cars, bicycles, in monocular images. These obstacles are then reported back to the ground vehicle.
